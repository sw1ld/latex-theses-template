\chapter{Hauptteil}
\label{cha:haupteil}

\section{Listen}
\label{sec:listen}

\textbf{Liste ohne Reihenfolge}
\begin{itemize}
    \item Alle Listenpunkte sind gleich
    \item Die Reihenfolge kann leicht geändert werden
    \item[\#] Explizit anderes Listenpunkt-Zeichen
    \begin{itemize}[*]
        \item Alle Listenpunkte haben ...
        \item das Zeichen * anstelle eines Punktes
        \item dieses Verhalten kann übrigens auch global in der config definiert werden.
    \end{itemize}
    \item Das ist der letzte Punkt
\end{itemize}

\textbf{Liste mit Reihenfolge}
\begin{enumerate}
    \item Erstes Argument - wichtigste Priorität
    \item Deshalb spielt die Reihenfolge eine wichtige Rolle
    \begin{enumerate}
        \item Liste innerhalb einer Liste
        \item Zweiter Unterpunkt
        \begin{itemize}
            \item[\textasciitilde] Punkt
            \item[$\to$] Mathematischer Operator als Zusammenfassung
        \end{itemize}
    \end{enumerate}
    \item Letzter Punkt
\end{enumerate}


\section{Tabellen}
\label{sec:tabellen}

\begin{table}[!htb]
    \centering
    \setlength{\colw}{0.15\textwidth-2\tabcolsep}    % 0.15 = Prozentsatz der Zellbreite für die Spalten

    \begin{tabular}{L{0.39\textwidth-2\tabcolsep}||C{\colw}|C{\colw}|C{\colw}|C{\colw}}
        & \textbf{A} & \textbf{B} & \textbf{C mit zwei Zeilen} & \textbf{D} \\
        \hhline{=::====} Kriterium 1 &
        \xmark & \cmark & \cmark & \cmark \\
        \hline Kriterium 2 mit Beschreibung über zwei Zeilen & Sehr gut & Gut & Schlecht & Schlecht \\
        \hline Kriterium 3 & Sehr gut & Gut & Schlecht & Schlecht \\
        \hline Kriterium 4 & Hoch & Sehr hoch & Gering & Keiner \\
    \end{tabular}
    \caption{Tabellenbeschreibung}
    \label{tab:tabellenbeschreibung}
\end{table}


\section{Abbildungen}
\label{sec:abbildungen}

\begin{figure}[htb]
    \centering
    \includegraphics[width=.3\textwidth]{bilder/logo.png}
    %\missingfigure{This is a missing figure}
    \caption{Testabbildung}
    \label{fig:testabbiludng}
\end{figure}

siehe abbildung \ref{fig:testabbiludng}

\begin{figure}[htb]
    \centering
    \begin{tikzpicture}[domain=0:4]
        \draw[very thin,color=gray] (-0.1,-1.1) grid (3.9,3.9);
        \draw[->] (-0.2,0) -- (4.2,0) node[right] {$x$};
        \draw[->] (0,-1.2) -- (0,4.2) node[above] {$f(x)$};
        \draw[color=red]    plot (\x,\x)             node[right] {$f(x) =x$};
        \draw[color=blue]   plot (\x,{sin(\x r)})    node[right] {$f(x) = \sin x$};
        \draw[color=orange] plot (\x,{0.05*exp(\x)}) node[right] {$f(x) = \frac{1}{20} \mathrm e^x$};
    \end{tikzpicture}
    \caption{Drawn picture}
    \label{fig:test-tikz}
\end{figure}



\section{Zitieren und Referenzieren}
\label{sec:zitieren-und-referenzieren}

Hier ist ein \textit{schlechtes Beispiel} für die Referenzierung von Tabelle 3.1, Abbildung 4.2 und Überschrift 1.

Dies ist ein \textbf{GUTES Beispiel} für die Referenzierung von Tabelle \ref{tab:tabellenbeschreibung}, Abbildung \ref{fig:testabbiludng} und Überschrift \ref{sec:einfuehrung}.

Dies ist beschrieben in \cite{7516158}. Man kann auch mehrere Referenzen auf einmal angeben \cite{siriwardena_advanced_2014, levin_restful_2016}.

\cite{7516158} hat die gleiche Referenz wie zuvor.

Es können auch spezifische Seitenzahlen angegeben werden \cite[S. 17f.]{rfc:6838}.


\section{Quellcode und Formeln}
\label{sec:quellcode-und-formeln}

\begin{equation}
    \int_0^\infty e^{-x^2} dx=\frac{\sqrt{\pi}}{2}
    \label{eq:test-formula}
\end{equation}

Dies ist eine einfache Art zur Erstellung von Formeln \textit{E=mc\textsuperscript{2}} weil es sich nicht um einen mathematischen Ausdruck handelt.

\begin{lstlisting}[caption={Java Code Example}, captionpos=b, label={lst:example-java}, language=Java]
// Hello.java
import javax.swing.JApplet;
import java.awt.Graphics;

public class Hello extends JApplet {
    public void paintComponent(Graphics g) {
        g.drawString("Hello, world!", 65, 95);
    }
}
\end{lstlisting}

Damit das Quellcodeverzeichnis erstellt wird und im Inhaltsverzeichnis auftaucht, muss in der main.tex die entsprechenden Zeilen einkommentiert werden!


\section{Acronyme und Glossar}
\label{sec:acronym-und-glossar}

Dies ist der \textit{falsche Weg} Acronyme zu verwenden:
\begin{itemize}
    \item Normal: Advanced Encryption Standard (AES)
    \item Kurze Form: AES
    \item Lange Form: Advanced Encryption
    \item Volle Form: Advanced Encryption Standard (AES)
    \item Plural: Advanced Encryption Sandards (AESs)
\end{itemize}

Dies ist der \textbf{richtige Weg} Acronyme zu verwenden:
\begin{itemize}
    \item Normal: \ac{AES}
    \item Using an acroynm twice: \ac{AES}
    \item Kurze Form: \acs{AES}
    \item Lange Form: \acl{AES}
    \item Volle Form: \acf{AES}
    \item Plural: \acfp{AES} (Vorsicht bei der Verwendung im Deutschen! LaTeX hängt nur ein 's' an das Wort an)
\end{itemize}

Acronyme werden nur im Abkürzungsverzeichnis angezeigt, wenn sie mindestens einmal verwendet wurden.

Ein Glossar ist eine Sammlung von detaillierten Beschreibungen in einem separaten Abschnitt der Arbeit. Glossareinträge können wie folgt verwendet werden: \gls{ReverseProxy}. \\
In dieser Vorlage werden alle Einträge im Kapitel \ref{cha:glossar} unabhängig der Verwendung dargestellt. Dieses Verhalten kann selbstverständlich auch geändert werden.
