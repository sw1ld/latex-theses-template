\chapter{Einführung}
\label{sec:einfuehrung}

\section{Unterkapitel}
\label{sec:unterkapitel}

Schreibe einen kurzen Satz mit Formatierungen. Hier ist eine Wort \textbf{fett} geschrieben, gefolgt von einem \textit{kursiven} Wort und abgeschlossen von einer \underline{Untersteichung}.\\
Zwei Backslashes sorgen für einen Zeilenumbruch. 

Leere Zeilen im Text sorgen für einen neuen Absatz.

Hier sind noch Sachen zu tun. \mytodo{Prof}{Frage an meinen Professor} Ich arbeite trotzdem weiter.

\textsc{Mögliche Formatierung für Überschriften} 

Bei Notizen an mich selbst kann ich \textit{mytodo} auch im Standardmodus verwenden \mytodo{}{Hier fehlt noch Text} 

\begin{center}
\huge Zentriert Text in größerer Schriftgröße
\end{center}

Es gibt einige Zeichen, die speziell maskiert werden müssen: 

\textbackslash \, wird für jeden Befehl benötigt

\& wird in Tabellen zur Unterteilung von Zellen verwendet

\$ Wird in Formlen verwendet

Leerzeichen wie "`\,"' or "`\quad"' können wie gezeigt dargestellt werden.

Explizite Leerzeichen, Seitenumbrüche, vertikale oder horizontale Platzhalter sind jedoch mit Vorsicht zu genießen, weil die Formatierung von LaTeX selbst organisiert werden sollte. Wenn die Struktur zu starr vorgegeben wird, können viele Vorteile nicht ausgespielt werden!

\section{Testkapitel}
\label{sec:testkapitel}

\subsection{Unter-Unterkapitel 1}
\label{sec:unter-unterkapitel1}

\subsection{Unter-Unterkapitel 2}
\label{sec:unter-unterkapitel2}

Für mehr Leerzeilen kann \textit{\textbackslash newline} verwendet werden: \newline \newline \newline Hier ist eine neue Zeile.

\vfill

Mit \textit{\textbackslash vfill} kann man in die letzte Zeile der Seite schreiben.
