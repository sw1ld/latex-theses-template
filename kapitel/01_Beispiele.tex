\chapter{Introduction}
\label{sec:introduction}

\section{Section One}
\label{sec:section-one}

siehe Kapitel \ref{sec:introduction}

Write a short sentence and do a little formatting. This is a \textbf{bold} word, followed by an \textit{italic} one, enclosed by an \underline{underlined} word. \\
Two backslashes break a line. 


stuff     to do \mytodo{Prof}{Question to my advisor} continue writing.

An empty line within the text creates a hole new passage. \textsc{This is a good font for Writing Headlines} 


If there is a problem with something latex-related stuff \mytodo{}{aks that question to your latex-supervisor ;-)} 

\begin{center}
\huge Centered with bigger text size
\end{center}

There are a few characters that need to be signed like \&, \$ \, or even have an own command like \textbackslash or spaces like "`\,"' or "`\quad"'.

\section{Testsection}
\label{sec:testsection}

Tabbed Example:

\qquad \textbackslash \, is used for every command

\qquad \& is used in tables to separete the contents

\qquad \$ is used for formulas

To get more than one empty line, use \textit{\textbackslash newline} \newline \newline \newline and continue writing.

\vfill
With \textit{\textbackslash vfill} you can write stuff in the bottom line of the page.


\chapter{Lists}
\label{cha:lists}


\textbf{Unordered}
\begin{itemize}
    \item All bullet points are the same
    \item You can easily change the order
    \item[\#] Explicit other sign
    \begin{itemize}[*]
        \item all items have ...
        \item this * instead of a dash 
    \end{itemize}
    \item This ist the last point
\end{itemize}

\textbf{Ordered}
\begin{enumerate}
    \item First argument is the most important one
    \item Therefore, the order matters
    \begin{enumerate}
        \item List in a list
        \item do stuff
        \begin{itemize}
            \item[\textasciitilde] Unordered List in an orderd list in an ordered list
            \item[$\to$] Mathematic operator as conclusion
        \end{itemize}
    \end{enumerate}
    \item Last bullet point
\end{enumerate}

\chapter{Table}
\label{cha:table}


\begin{table}[!htb]
    \centering
    \setlength{\colw}{0.15\textwidth-2\tabcolsep}    % 0.15 = Prozentsatz der Zellbreite für die Spalten
    
    \begin{tabular}{L{0.39\textwidth-2\tabcolsep}||C{\colw}|C{\colw}|C{\colw}|C{\colw}}
        & \textbf{A} & \textbf{B} & \textbf{C mit zwei Zeilen} & \textbf{D} \\
        \hhline{=::====} Kriterium 1 &
        \xmark & \cmark & \cmark & \cmark \\
        \hline Kriterium 2 mit Beschreibung über zwei Zeilen & Sehr gut & Gut & Schlecht & Schlecht \\
        \hline Kriterium 3 & Sehr gut & Gut & Schlecht & Schlecht \\
        \hline Kriterium 4 & Hoch & Sehr hoch & Gering & Keiner \\
    \end{tabular}
    \caption{Caption}
    \label{tab:caption-table-1}
\end{table}

\chapter{Figures}
\label{cha:figures}

\begin{figure}[htb]
    \centering
    \includegraphics[width=.5\textwidth]{bilder/logo.png}
    %\missingfigure{This is a missing figure}
    \caption{Test figure}
    \label{fig:test-figure}
\end{figure}

siehe abbildung \ref{fig:test-figure}

\begin{figure}[htb]
    \centering
    \begin{tikzpicture}[domain=0:4] 
        \draw[very thin,color=gray] (-0.1,-1.1) grid (3.9,3.9);
        \draw[->] (-0.2,0) -- (4.2,0) node[right] {$x$}; 
        \draw[->] (0,-1.2) -- (0,4.2) node[above] {$f(x)$};
        \draw[color=red]    plot (\x,\x)             node[right] {$f(x) =x$}; 
        \draw[color=blue]   plot (\x,{sin(\x r)})    node[right] {$f(x) = \sin x$}; 
        \draw[color=orange] plot (\x,{0.05*exp(\x)}) node[right] {$f(x) = \frac{1}{20} \mathrm e^x$};
    \end{tikzpicture}
    \caption{Drawn picture}
    \label{fig:test-tikz}
\end{figure}



\chapter{Citations and References}
\label{cha:citations-and-references}

This is a \textit{bad example} to reference Table 3.1, Figure 4.2 and Section 1.

This is a \textbf{GOOD example} to reference Table \ref{tab:caption-table-1}, Figure \ref{fig:test-figure} and Section \ref{sec:section-one}.

This is what is described in \cite{7516158}. Also you can reference \cite{siriwardena_advanced_2014, levin_restful_2016} in a single citation. 

\cite{7516158} will have the same number as before.

To be more specific you can cite a source with an optional pagenumber \cite[p. 17]{rfc:6838}.

\chapter{Acronyms and Glossary}
\label{sec:acronyms-and-glossary}

This is the \textit{wrong way} to use Acronyms: 
\begin{itemize}
    \item Normal Acronym: Advanced Encryption Standard (AES)
    \item Short Form: AES
    \item Long Form: Advanced Encryption
    \item Full Form: Advanced Encryption Standard (AES)
    \item Plural: Advanced Encryption Sandards (AESs)
\end{itemize}

This is the \textbf{correct way} to use Acronyms: 
\begin{itemize}
    \item Normal Acronym: \ac{AES}
    \item Using an acroynm twice: \ac{AES}
    \item Short Form: \acs{AES}
    \item Long Form: \acl{AES}
    \item Full Form: \acf{AES}
    \item Plural: \acfp{AES} (take care for other languages)
\end{itemize}

A glossary is a description for something, that is displayed on a spearate page. You can reference a glossary entry like this: \gls{ReverseProxy}. In this template every glossary entry gets displayed in Chapter \ref{cha:glossar} no matter if it is used in the text. However, this behaviour can easily be changed.


\chapter{Source Code and Formulas}
\label{chap:source-code-and-formulas}

\begin{equation} 
    \int_0^\infty e^{-x^2} dx=\frac{\sqrt{\pi}}{2} 
    \label{eq:test-formula}
\end{equation}

This is a bad choice for a formula \textit{E=mc\textsuperscript{2}} because it is not a mathematical expression.

\begin{lstlisting}[caption={Java Code Example}, captionpos=b, label={lst:example-java}, language=Java]
// Hello.java
import javax.swing.JApplet;
import java.awt.Graphics;

public class Hello extends JApplet {
    public void paintComponent(Graphics g) {
        g.drawString("Hello, world!", 65, 95);
    }    
}
\end{lstlisting}

To have a list of listings the command in main.tex (line 52-53) must be un-commented.
