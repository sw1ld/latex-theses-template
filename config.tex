\usepackage[
    automark,       % Kapitelangaben in Kopfzeile automatisch erstellen
    headsepline,    % Trennlinie unter Kopfzeile
    footsepline,
    ilines          % Trennlinie linksbündig ausrichten
]{scrpage2}

\usepackage[ngerman]{babel}
\usepackage[utf8]{inputenc}
\usepackage[T1]{fontenc}

\usepackage{setspace}
\onehalfspacing % Zeilenabstand 1,5 Zeilen

\usepackage{geometry}
\setlength{\topskip}{\ht\strutbox} % behebt Warnung von geometry
\geometry{left=38mm,right=27mm,top=25mm,bottom=45mm}

% Kopf- und Fußzeilen
\pagestyle{scrheadings}
% Kopf- und Fußzeile auch auf Kapitelanfangsseiten
\renewcommand*{\chapterpagestyle}{scrheadings} 
% Schriftform der Kopfzeile
\renewcommand{\headfont}{\normalfont}

% Kopfzeile
%\ihead{\large{\textsc{\titel}} \\[2ex] \textit{\headmark}}
\ihead{\large{} \\[2ex] \textit{\headmark}}
\chead{}
\ohead{\includegraphics[scale=0.4]{\logo} \hspace*{25mm}}   % optionales Logo der Universität

\setlength{\headheight}{21mm} % Höhe der Kopfzeile
% Kopfzeile über den Text hinaus verbreitern
\setheadwidth[0pt]{textwithmarginpar} 
\setheadsepline[text]{0.4pt} % Trennlinie unter Kopfzeile

% Fußzeile
\ifoot{\autor}
\cfoot{}
\ofoot{\pagemark}

\usepackage{courier}
\usepackage{relsize} 
\usepackage[dvips,final]{graphicx}
\usepackage{amsmath}
\usepackage{amsfonts}
\usepackage{longtable}
\newlength\colw
\usepackage[section]{placeins}

\usepackage{seqsplit}       % breaks text at end of line
\usepackage[hyphens]{url}   % breaks URLs

\usepackage{eurosym}

\usepackage{booktabs}
\usepackage[table,xcdraw]{xcolor}
\usepackage{rotating,tabularx}

\usepackage{tablefootnote}

\usepackage{hhline}
\usepackage{array}
\newcolumntype{C}[1]{>{\centering\let\newline\\\arraybackslash\hspace{0pt}}m{#1}}   % center
\newcolumntype{L}[1]{>{\raggedright\let\newline\\\arraybackslash\hspace{0pt}}m{#1}} % left
\newcolumntype{R}[1]{>{\raggedleft\let\newline\\\arraybackslash\hspace{0pt}}m{#1}}  % right
\newcolumntype{w}[1]{>{\raggedleft\hspace{0pt}}p{#1}}


\usepackage{paralist} % Formatierung von Listen ändern
\usepackage{todonotes}
% usage: \todod{Adressat}{Frage}
\newcommand{\mytodo}[2]{
    \IfEqCase{#1}{
        {Prof}{\todo[color=orange]{#1}{\textit{#2}}}
        {cite}{\todo[color=blue]{#1}{}}
        {newpage}{\todo[color=red]{#1}{}}
        {}{\todo[color=green]{#1}{\textit{#2}}}
    }[\PackageError{mytodo}{Undefined option to todod: #1}{}]
}


\usepackage[printonlyused]{acronym}

\usepackage{chronology}
\usepackage{tikz}
\usepackage{pgfplots}
\usepackage{numprint}
\pgfplotsset{compat=1.14} 
\usepackage{mathptmx}
\usepackage{calc}
\usepackage{xcolor} 

\usepackage[toc]{glossaries}
\loadglsentries{einzelseiten/Glossar}
\makeglossaries
\glsaddall{}

\usepackage{pifont}
\newcommand{\cmark}{\textcolor[rgb]{0,0.6,0}{\ding{51}}}
\newcommand{\xmark}{\textcolor{red}{\ding{55}}}

\newcommand*{\TakeFourierOrnament}[1]{{\fontencoding{U}\fontfamily{futs}\selectfont\char#1}}
\newcommand*{\danger}{\textcolor[RGB]{255,200,0}{\TakeFourierOrnament{66}}}

\frenchspacing % erzeugt ein wenig mehr Platz hinter einem Punkt


\usepackage{pdfpages}

% Schusterjungen und Hurenkinder vermeiden
\clubpenalty = 10000
\widowpenalty = 10000 
\displaywidowpenalty = 10000

\usepackage{floatflt}

\usepackage{listings}
\definecolor{colBackground}{RGB}{245, 245, 245}
\definecolor{colKeys}{RGB}{0, 0, 255}
\definecolor{colIdentifier}{RGB}{0, 0, 0}
\definecolor{colComments}{RGB}{40, 220, 0}
\definecolor{colString}{RGB}{0, 150, 0}
\lstset{
    float=hbp,
    basicstyle=\ttfamily\color{black}\small\smaller,
    identifierstyle=\color{colIdentifier},
    keywordstyle=\color{colKeys},
    stringstyle=\color{colString},
    commentstyle=\color{colComments},
    columns=flexible,
    tabsize=2,
    frame=single,
    extendedchars=true,
    showspaces=false,
    showstringspaces=false,
    numbers=left,
    numberstyle=\tiny,
    numbersep=5pt,
    breaklines=true,
    backgroundcolor=\color{colBackground},
    emph={square}, 
    emphstyle=\color{red}, 
    emph={[2]root,base}, 
    emphstyle={[2]\color{blue}},
    breakautoindent=true
}

\lstdefinelanguage{json}{
    %string=[s]{"}{"},
    %stringstyle=\color{blue},
    %comment=[l]{:},
    %commentstyle=\color{black},
}

\usepackage[
    bookmarks,
    bookmarksopen=true,
    colorlinks=true,
    linkcolor=black,
    anchorcolor=black,
    citecolor=black,
    filecolor=black,
    menucolor=black,
    urlcolor=black, 
    %backref,
    plainpages=false,
    pdfpagelabels,
    hypertexnames=false
]{hyperref}


% Abkürzungen mit korrektem Leerraum 
\newcommand{\ua}{\mbox{u.\,a.\ }}
\newcommand{\zB}{\mbox{z.\,B.\ }}
\newcommand{\vgl}{vgl.\ }
\newcommand{\bzw}{bzw.\ }
\newcommand{\evtl}{evtl.\ }
\newcommand{\ggf}{ggf.\ }
\newcommand{\idR}{i.\,d.\,R.\ }

\newcommand{\bs}{$\backslash$}
\newcommand{\arrow}{$\to$}

% Listenelement mit fetter Überschrift und Zeilenumbruch -> \itemd{Überschrift}
\newcommand{\itemd}[1]{\item{\textbf{#1}}\\}

\newcommand{\textcour}[1]{
    \begin{footnotesize}
        \texttt{#1} 
    \end{footnotesize}
}

% Tabelleninhalte werden grundsätzlich in kleinerer Schriftgröße erstellt
\usepackage{etoolbox}
\AtBeginEnvironment{tabular}{\footnotesize}

% Gantt Diagramm
\usepackage{pgfgantt}
\definecolor{textcol}{RGB}{48,48,48}
\definecolor{completed1}{RGB}{127,127,127}
\definecolor{completed2}{RGB}{217,217,217}
\definecolor{todo1}{RGB}{13,110,161}
\definecolor{todo2}{RGB}{21,171,253}
